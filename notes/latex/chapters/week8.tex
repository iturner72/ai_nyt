\newpage
%%%%%%%%%%%%%%%%%%%%%%%%%%%%%%%%%%%%%%%%%%%%%%%%%%%%%%%%%%%%%%%%%%%%%%%%%%%%%%%
\section{Week 8?}
%%%%%%%%%%%%%%%%%%%%%%%%%%%%%%%%%%%%%%%%%%%%%%%%%%%%%%%%%%%%%%%%%%%%%%%%%%%%%%%
\subsection*{Sunday, 06/10/2024}
\begin{itemize}
    \item we might be back (leptos rewrite). notes from testing existing hubble
        node routes with leptos components.
    \item this is not live yet since it has very limited funtionality.
    \item \texttt{fetch_username} checks if \texttt{lead_usernames} already 
        has the \texttt{fid}. if not, it fetches the username and updates the 
        state.
    \item added \texttt{ongoing_requests} to track fetches and avoid multiple 
        requests for the same \texttt{fid}.
    \item Used \texttt{create_effect} to manage side effects, ensuring 
        usernames are fetched only once.
    \item Used \texttt{spawn_local} for async tasks to keep the main thread 
        non-blocking.
    \item \texttt{Signal} and \texttt{set_signal} handle reactive state in the 
        \texttt{Channels} component, making sure the UI updates when data 
        changes.
    \item \texttt{HashSet} tracks \texttt{ongoing_requests}, preventing 
        duplicate fetches and redundant state updates.
    \item The component dynamically displays usernames from the updated state.
\end{itemize}
